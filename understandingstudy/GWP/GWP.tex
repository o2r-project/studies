%% Copernicus Publications Manuscript Preparation Template for LaTeX Submissions
%% ---------------------------------
%% This template should be used for copernicus.cls
%% The class file and some style files are bundled in the Copernicus Latex Package, which can be downloaded from the different journal webpages.
%% For further assistance please contact Copernicus Publications at: production@copernicus.org
%% https://publications.copernicus.org/for_authors/manuscript_preparation.html

%% copernicus_rticles_template (flag for rticles template detection - do not remove!)

%% Please use the following documentclass and journal abbreviations for discussion papers and final revised papers.

%% 2-column papers and discussion papers
\documentclass[gc, manuscript]{copernicus}



%% Journal abbreviations (please use the same for discussion papers and final revised papers)


% Advances in Geosciences (adgeo)
% Advances in Radio Science (ars)
% Advances in Science and Research (asr)
% Advances in Statistical Climatology, Meteorology and Oceanography (ascmo)
% Annales Geophysicae (angeo)
% Archives Animal Breeding (aab)
% ASTRA Proceedings (ap)
% Atmospheric Chemistry and Physics (acp)
% Atmospheric Measurement Techniques (amt)
% Biogeosciences (bg)
% Climate of the Past (cp)
% DEUQUA Special Publications (deuquasp)
% Drinking Water Engineering and Science (dwes)
% Earth Surface Dynamics (esurf)
% Earth System Dynamics (esd)
% Earth System Science Data (essd)
% E&G Quaternary Science Journal (egqsj)
% Fossil Record (fr)
% Geochronology (gchron)
% Geographica Helvetica (gh)
% Geoscience Communication (gc)
% Geoscientific Instrumentation, Methods and Data Systems (gi)
% Geoscientific Model Development (gmd)
% History of Geo- and Space Sciences (hgss)
% Hydrology and Earth System Sciences (hess)
% Journal of Micropalaeontology (jm)
% Journal of Sensors and Sensor Systems (jsss)
% Mechanical Sciences (ms)
% Natural Hazards and Earth System Sciences (nhess)
% Nonlinear Processes in Geophysics (npg)
% Ocean Science (os)
% Primate Biology (pb)
% Proceedings of the International Association of Hydrological Sciences (piahs)
% Scientific Drilling (sd)
% SOIL (soil)
% Solid Earth (se)
% The Cryosphere (tc)
% Web Ecology (we)
% Wind Energy Science (wes)


%% \usepackage commands included in the copernicus.cls:
%\usepackage[german, english]{babel}
%\usepackage{tabularx}
%\usepackage{cancel}
%\usepackage{multirow}
%\usepackage{supertabular}
%\usepackage{algorithmic}
%\usepackage{algorithm}
%\usepackage{amsthm}
%\usepackage{float}
%\usepackage{subfig}
%\usepackage{rotating}

% The "Technical instructions for LaTex" by Copernicus require _not_ to insert any additional packages.
%


\begin{document}

\title{A quantitative approach to evaluating the GWP timescale through implicit
discount rates}


\Author[]{}{}



%% The [] brackets identify the author with the corresponding affiliation. 1, 2, 3, etc. should be inserted.



\runningtitle{INSYDE}

\runningauthor{Dottori}





\received{}
\pubdiscuss{} %% only important for two-stage journals
\revised{}
\accepted{}
\published{}

%% These dates will be inserted by Copernicus Publications during the typesetting process.


\firstpage{1}

\maketitle






\section{Introduction}

The global warming potential (GWP) is the primary metric used to assess
the equivalency of emissions of different greenhouse gases (GHGs). This
primacy was established soon after its development in 1990 due to its
early use by the WMO (1992) and UNFCCC (1995). However, despite the
political acceptance, the GWP has also been a source of controversy and
criticism: that radiative forcing as a measure of impact is not as
relevant as temperature or damages; that the assumption of constant
future GHG concentrations is unrealistic; that discounting is preferred
to a constant time period of integration; disagreements about the choice
of time horizon in the absence of discounting; that dynamic approaches
would lead to a more optimal resource allocation over time; that the GWP
does not account for non-climatic effects such as carbon fertilization
or ozone produced by methane; and that pulses of emissions are less
relevant than streams of emissions. Including these complicating factors
would make the metric less simple and transparent and would require
reaching a consensus regarding appropriate parameter values and model
choices. The simplicity of the calculation of the GWP is one of the
reasons for its widespread use. In this paper, we focus on the choice of
time horizon in the GWP that can reflect decision-maker values, but for
which additional clarity regarding the implications of the time horizon
could be useful. We also investigate the extent to which the choice of
time horizon can incorporate many of the complexities of assessing the
impacts described in the previous paragraph. The 100-year time horizon
of the GWP (GWP100) is the time horizon most commonly used in many
venues, for example in the Kyoto Protocol, perhaps because it was the
middle value of the three time horizons (20, 100, and 500 years)
analyzed in the IPCC First Assessment Report. However, the 100-year time
horizon has been described by some as arbitrary. The IPCC AR5 stated
that ``{[}t{]}here is no scientific argument for selecting 100 years
compared with other choices''. The WMO (1992) assessment has provided a
justifications for the 100-year time horizon, stating that ``the GWPs
evaluated over the 100-year period appear generally to provide a
balanced representation of the various time horizons for climate
response''. Recently, some researchers and NGOs have been promoting more
emphasis on shorter time horizons, such as 20 years, which would
highlight the role of short-lived climate forcers such as
CH\textsubscript{4}. These studies each have different nuances regarding
their recommendations -- for example, pairing the GWP100 with the GWP20
to reflect both long-term and near-term climate impacts. It is plausible
that more consideration of short-term metrics would result in policy
that weights near-term impacts more heavily. In contrast, some
governments have suggested the use of the 100-year global temperature
change potential (GTP) based on the greater physical relevance of
temperature in comparison to forcing, in effect downplaying the role of
the same short-lived climate forcer. Therefore, the question of
timescale remains an area of active debate. We argue that more focus on
quantitative justifications for timescales within the GWP structure
would be of value, as differentiated from qualitative justifications
such as a need for urgency to avoid tipping points. This paper provides
a needed quantification and analysis of the implications of different
GWP time horizons. We follow the lead of economists who have proposed
that the appropriate comparison for different options for GHG emissions
policies is between the net present discounted marginal damages. This
paper reframes and clarifies key issues and presents a framework for
better understanding how different timescales can be reconciled with how
the future is valued. The paper focuses on CO\textsubscript{2} and
CH\textsubscript{4} as the two most important anthropogenic contributors
to current warming, but the methodology is applicable to emissions of
other gases, and sensitivity analyses consider N\textsubscript{2}O.

\section{Methods}

The general approach taken in this paper is to calculate the impact of a
pulse of emissions of either CO\textsubscript{2} or CH\textsubscript{4}
in the first year of simulation on a series of climatic variables. The
first step is to calculate the perturbation of atmospheric
concentrations over a baseline scenario. The concentration perturbation
is transformed into a change in the global radiative forcing balance.
The radiative forcing perturbation over time is used to calculate the
impact on temperature and then damages due to that temperature change.
Discount rates are then applied to these impacts to determine the net
present value of the impacts. The details of these calculations are
described here.

\textbf{Concentration:}

\section{Results}

\subsection{\texorpdfstring{Evaluating the climate effects of an
emission pulse of
CH\textsubscript{4}}{Evaluating the climate effects of an emission pulse of CH}}

The analysis starts by calculating the climate effects of an emission
pulse of CH4. We introduce an emission pulse of 28.3MT in 2011 (yielding
a 10 ppb increase in CH4 concentration in the initial year) applied on
top of the GHG concentrations of Representative Concentration Pathway
(RCP) 6.0. Figure 1 shows the changes in radiative forcing (RF; a),
temperature (T; b), damages (c), and damages discounted at a 3\% rate
(d) out to the year 2300 resulting from such a pulse. Figure 1 relies on
calculations that use central estimates of the uncertain parameters, as
discussed in the Methods section. While the graph is truncated at 2300,
the calculations used in this paper extend to 2500. The impacts of an
emission pulse of CO2 are also shown using 24.8 times the mass of the
CH4 pulse (this factor is chosen to create equivalent integrated damages
over the full time period when discounted at 3\% as shown in Fig. 1d).
Figure 1a and b demonstrate the trade-offs between near- term and
long-term impacts when assigning equivalency to emission pulses of
different lifetimes. After 100 years, the radiative forcing effects of
the CH4 pulse decay to 0.04\% of the initial forcing in the year of the
emission pulse, and the temperature effects decay to 4\% of the peak
temperature (reached 10 years after the pulse). In contrast, after 100
years the radiative forcing effects of the CO2 pulse decay to 22\% of
the initial forcing, and the temperature effects decay to 51\% of the
CO2 peak temperature (reached 18 years after the pulse). The immediacy
of the temperature effects for the CH4 pulse creates larger damages in
both overall and discounted dollar terms for the first 42 years. After
43 years, the sustained CO2 effects overtake the CH4 effects. With a
different discount rate, a different factor would have been used to
calculate the CO2 mass used for the CO2 pulse, which would change the
crossing point for damages -- a higher discount rate would require a
larger CO2 equivalent pulse relative to the CH4 pulse and therefore an
earlier crossing point (and vice versa). Figure 1c demonstrates the
dramatic increase in damage over time due to the relationship of damage
to economic growth. In the case of CH4, damage peaks in 2032 and
declines until 2080 as a result of the short lifetime of the gas. The
increase in damages after 2080 is due to the component of the
temperature response function that includes a 409-year timescale decay
rate such that after 100 years the decrease in the DELTAT2 component of
the damage equation is about 0.5\%/year, and because that decay rate is
slower than the rate of GDP growth, net damages grow. Figure 1d
demonstrates the dramatic decrease in future damages when applying a
constant discount rate. Taken as a whole, these four figures demonstrate
the trade-offs required when attempting to create equivalences for
emissions of gases with very different lifetimes.

\subsection{Sensitivity analysis}

Figure 2 shows the median, interquartile, interdecile, and extremes of
the equivalent GWP time horizon corresponding to a given discount rate
from a sensitivity analysis. The uncertainty was calculated assuming
equal likelihood of each of the 972 combinations of all of the parameter
choices used in this paper: four RCPs, three climate sensitivities,
three damage exponents, three forcing imbalance options, three
temperature offsets, and three GDP growth rates. The ranges chosen for
each parameter are described in the Methods section. The parameters with
the largest effect on the uncertainty of the calculated GWP (at a
discount rate of 3 \%) are the rate of GDP growth and the damage
exponent (see Table 1). For these six parameters, the choices that lead
to larger damages from CH4 relative to CO2 are a low GDP growth, a low
damage exponent, a low-emissions scenario, a higher temperature offset
(e.g., assuming that damages are a function of warming from
preindustrial, not warming from present day), a lower climate
sensitivity, and a higher current forcing imbalance. The general trend
is that the more that damages are expected to grow in the future (e.g.,
high GDP growth, damage exponent, or emissions scenario), the longer the
equivalent timescale is for a given discount rate. While CO2 and CH4 are
the largest contributors to climate change (as evaluated by
contributions of historical emissions to present-day radiative forcing
as in Table 8.SM.6 in the IPCC and by the magnitude of present-day
emissions as evaluated by the standard GWP100), it is also informative
to evaluate emissions of other gases with these techniques. Table 2
shows five gases and their atmospheric lifetimes. For each gas, an
``optimal'' GWP timescale was calculated that would replicate the ratio
of net present damage of that gas to CO2 at a discount rate of 3 \%. The
ratio of the GWP100 and the GWP20 to that optimal damage ratio is also
shown. For longer-lived gases (e.g., N2O and HFC-23), there is no
integration time period that can produce a ratio as large as the
calculated damage ratio at a discount rate of 3 \%. For these gases, we
list the timescale that yields the maximum possible ratio and note that
the GWP for longer-lived gases is fairly insensitive to timescale
(further comparisons of non-CO2 gases are presented in the Supplement).
This table shows that at a discount rate of 3\% and as evaluated using
net present damage ratios, the use of a 100-year timescale is consistent
(interquartile range) with the optimal timescale / damage ratios for
methane. For gases with lifetimes in centuries, the GWP at any timescale
undervalues these gases, but the magnitude of that undervaluation is
somewhat insensitive to the choice of timescale. For the longest-lived
gases, the GWP also undervalues reductions in these gases, but the
longer the timescale the better the match.







%%%%%%%%%%%%%%%%%%%%%%%%%%%%%%%%%%%%%%%%%%
%% optional

%%%%%%%%%%%%%%%%%%%%%%%%%%%%%%%%%%%%%%%%%%

%%%%%%%%%%%%%%%%%%%%%%%%%%%%%%%%%%%%%%%%%%

%%%%%%%%%%%%%%%%%%%%%%%%%%%%%%%%%%%%%%%%%%
\competinginterests{} %% this section is mandatory even if you declare that no competing interests are present

%%%%%%%%%%%%%%%%%%%%%%%%%%%%%%%%%%%%%%%%%%

%%%%%%%%%%%%%%%%%%%%%%%%%%%%%%%%%%%%%%%%%%

%% REFERENCES
%% DN: pre-configured to BibTeX for rticles

%% The reference list is compiled as follows:
%%
%% \begin{thebibliography}{}
%%
%% \bibitem[AUTHOR(YEAR)]{LABEL1}
%% REFERENCE 1
%%
%% \bibitem[AUTHOR(YEAR)]{LABEL2}
%% REFERENCE 2
%%
%% \end{thebibliography}

%% Since the Copernicus LaTeX package includes the BibTeX style file copernicus.bst,
%% authors experienced with BibTeX only have to include the following two lines:
%%
\bibliographystyle{copernicus}
\bibliography{}
%%
%% URLs and DOIs can be entered in your BibTeX file as:
%%
%% URL = {http://www.xyz.org/~jones/idx_g.htm}
%% DOI = {10.5194/xyz}


%% LITERATURE CITATIONS
%%
%% command                        & example result
%% \citet{jones90}|               & Jones et al. (1990)
%% \citep{jones90}|               & (Jones et al., 1990)
%% \citep{jones90,jones93}|       & (Jones et al., 1990, 1993)
%% \citep[p.~32]{jones90}|        & (Jones et al., 1990, p.~32)
%% \citep[e.g.,][]{jones90}|      & (e.g., Jones et al., 1990)
%% \citep[e.g.,][p.~32]{jones90}| & (e.g., Jones et al., 1990, p.~32)
%% \citeauthor{jones90}|          & Jones et al.
%% \citeyear{jones90}|            & 1990

\end{document}
